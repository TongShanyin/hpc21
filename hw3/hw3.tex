\documentclass[10pt,a4paper]{article}
\usepackage[centertags]{amsmath}
\usepackage{amsfonts,amssymb, amsthm}
\usepackage{hyperref}
\usepackage{comment}
\usepackage[shortlabels]{enumitem}
\usepackage{bm}

\usepackage{cite,graphicx,color}
%\usepackage{fourier}
\usepackage[margin=1.5in]{geometry}
\usepackage{enumitem}
\usepackage{bbm}

\usepackage{tikz,pgfplots}

\usepackage{listings}
\usepackage{xcolor}

\definecolor{codegreen}{rgb}{0,0.6,0}
\definecolor{codegray}{rgb}{0.5,0.5,0.5}
\definecolor{codepurple}{rgb}{0.58,0,0.82}
\definecolor{backcolour}{rgb}{0.95,0.95,0.92}

\lstdefinestyle{mystyle}{
	backgroundcolor=\color{backcolour},   
	commentstyle=\color{codegreen},
	keywordstyle=\color{magenta},
	numberstyle=\tiny\color{codegray},
	stringstyle=\color{codepurple},
	basicstyle=\ttfamily\footnotesize,
	breakatwhitespace=false,         
	breaklines=true,                 
	captionpos=b,                    
	keepspaces=true,                 
	numbers=left,                    
	numbersep=5pt,                  
	showspaces=false,                
	showstringspaces=false,
	showtabs=false,                  
	tabsize=2
}

\lstset{style=mystyle}

\usepackage{mathtools}
%\mathtoolsset{showonlyrefs} % only show no. of refered eqs

\usepackage{cleveref}

\textheight 8.5in

\newtheorem{theorem}{Theorem}
\newtheorem{assumption}{Assumption}
\newtheorem{example}{Example}
\newtheorem{proposition}{Proposition}

\newtheoremstyle{dotlessP}{}{}{\color{blue!50!black}}{}{\color{blue}\bfseries}{}{ }{}
\theoremstyle{dotlessP}
\newtheorem{question}{Question}



\def\VV{\mathbb{V}}
\def\EE{\mathbb{E}}
\def\PP{\mathbb{P}}
\def\RR{\mathbb{R}}
\newcommand{\mD}{\mathcal{D}}
\newcommand{\mF}{F}%{\mathcal{F}}

\DeclareMathOperator{\sgn}{sgn}
%\DeclareMathOperator{\erf}{erf}
\DeclareMathOperator{\erfc}{erfc}
\DeclareRobustCommand{\argmin}{\operatorname*{argmin}}
\DeclareRobustCommand{\arginf}{\operatorname*{arginf}}

\def\EE{\mathbb{E}}\def\PP{\mathbb{P}}
\def\NN{\mathbb{N}}\def\RR{\mathbb{R}}\def\ZZ{\mathbb{Z}}



\def\<{\left\langle} \def\>{\right\rangle}







\newcommand{\emphasis}[1]{\textcolor{red!80!black}{#1}}
\newcommand{\shanyin}[1]{\textcolor{blue!80!black}{#1}}

% ****************************
\begin{document}


\title{High performance computing HW3}
\author{Shanyin Tong, st3255@nyu.edu}

\maketitle

\section{OpenMP warm-up}
\begin{enumerate}[(a)]
	\item Since there are two chunks,  OpenMP divides iterations into chunks that are approximately equal in size, then each chunks is about size $n/2$. 
	
	For the first for-loop,  the execution for the first part is $$1+2+\ldots+\frac{n}{2}=\frac{1}{2}(1+\frac{n}{2})\frac{n}{2}=\frac{n(n+2)}{8},$$ the second part is 
	$$ (\frac{n}{2}+1) + \ldots + (n-1) = \frac{1}{2}( \frac{n}{2}+1+n-1)(\frac{n}{2}-1)=\frac{3n(n-2)}{8}.$$ Assume $n$ is large, the total execution time is 
	$$\max\{\frac{n(n+2)}{8}, \frac{3n(n-2)}{8}\}=  \frac{3n(n-2)}{8},$$ the wait time is $$\frac{3n(n-2)}{8} - \frac{n(n+2)}{8} = \frac{n^2-2n}{4} .$$
	
	For the second for-loop, the execution for the first part is $$(n-1)+(n-2)+\ldots+(n-\frac{n}{2})=\frac{1}{2}(n-1+n-\frac{n}{2})\frac{n}{2}=\frac{n(3n-2)}{8},$$ the second part is 
	$$ (n-\frac{n}{2}-1 )+ \ldots +(n- n+1) = \frac{1}{2}( n-\frac{n}{2}-1+n-n+1)(\frac{n}{2}-1)=\frac{n(n-2)}{8}.$$ Assume $n$ is large, the total execution time is 
	$$\max\{\frac{n(3n-2)}{8}, \frac{n(n-2)}{8}\}=  \frac{n(3n-2)}{8},$$ the wait time is $$\frac{n(3n-2)}{8} - \frac{n(n-2)}{8} = \frac{n^2}{4} .$$
	
	Thus, the total time each thread spend to execute the parallel region is $\frac{3n(n-2)}{8}+\frac{n(3n-2)}{8}=\frac{3n^2-4n}{4}$ milliseconds, the total wait time is $\frac{n^2-2n}{4}+\frac{n^2}{4}=\frac{n^2-n}{2}$ milliseconds.
	
	
	\item Using \texttt{schedule(static, 1)} means the chunk size is 1. The first thread execute iterations $1, 3, \ldots, n-1$, and the second thread execute iterations $2, 4, \ldots, n-2$. So for the first loop, thread 1 takes $1+3+\ldots+n-1=\frac{n^2}{4}$ milliseconds, thread 2 takes $\frac{n(n-2)}{4}$ milliseconds, so the total execution time is $\frac{n^2}{4}$ milliseconds, wait time is $\frac{n}{2}$ milliseconds. For the second loop, the situation is the same as the first loop. So the total execution time for each thread is $\frac{n^2}{2}$ milliseconds, smaller than (a). 
	
	\item By using \texttt{schedule(dynamic, 1)}, since OpenMP dynamically distributes the iterations during the runtime, when thread 1 takes iteration 1, thread 2 take iteration 2, but thread 1 execution time is shorter than thread 2, so thread 1 will take iteration 3... So thread 1 will take iterations $1, 3, \ldots, n-1$ and thread 2 will take iterations $2, 4, \ldots, n-2$, which is same as (b), so it will not improve the run time.
	
	\item We can use \texttt{nowait} to eliminate the wait time. When using \texttt{nowait}, in case (a), thread 1 uses $\frac{n(n+2)}{8}+\frac{n(3n-2)}{8}=\frac{n^2}{2}$ milliseconds, thread 2 uses $\frac{3n(n-2)}{8}+\frac{n(n-2)}{8}=\frac{n^2-2n}{2}$ milliseconds. In case (b), thread 1 uses $\frac{n^2}{2}$, thread 2 uses $\frac{n^2-2n}{2}$ milliseconds. In case(c), both thread uses $\frac{n^2-n}{2}$ milliseconds.
	
\end{enumerate}

\end{document}