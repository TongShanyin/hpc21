\documentclass[10pt,a4paper]{article}
\usepackage[centertags]{amsmath}
\usepackage{amsfonts,amssymb, amsthm}
\usepackage{hyperref}
\usepackage{comment}
\usepackage[shortlabels]{enumitem}
\usepackage{bm}

\usepackage{cite,graphicx,color}
%\usepackage{fourier}
\usepackage[margin=1.5in]{geometry}
\usepackage{enumitem}
\usepackage{bbm}

\usepackage{tikz,pgfplots}

\usepackage{listings}
\usepackage{xcolor}

\definecolor{codegreen}{rgb}{0,0.6,0}
\definecolor{codegray}{rgb}{0.5,0.5,0.5}
\definecolor{codepurple}{rgb}{0.58,0,0.82}
\definecolor{backcolour}{rgb}{0.95,0.95,0.92}

\lstdefinestyle{mystyle}{
	backgroundcolor=\color{backcolour},   
	commentstyle=\color{codegreen},
	keywordstyle=\color{magenta},
	numberstyle=\tiny\color{codegray},
	stringstyle=\color{codepurple},
	basicstyle=\ttfamily\footnotesize,
	breakatwhitespace=false,         
	breaklines=true,                 
	captionpos=b,                    
	keepspaces=true,                 
	numbers=left,                    
	numbersep=5pt,                  
	showspaces=false,                
	showstringspaces=false,
	showtabs=false,                  
	tabsize=2
}

\lstset{style=mystyle}

\usepackage{mathtools}
%\mathtoolsset{showonlyrefs} % only show no. of refered eqs

\usepackage{cleveref}

\textheight 8.5in

\newtheorem{theorem}{Theorem}
\newtheorem{assumption}{Assumption}
\newtheorem{example}{Example}
\newtheorem{proposition}{Proposition}

\newtheoremstyle{dotlessP}{}{}{\color{blue!50!black}}{}{\color{blue}\bfseries}{}{ }{}
\theoremstyle{dotlessP}
\newtheorem{question}{Question}



\def\VV{\mathbb{V}}
\def\EE{\mathbb{E}}
\def\PP{\mathbb{P}}
\def\RR{\mathbb{R}}
\newcommand{\mD}{\mathcal{D}}
\newcommand{\mF}{F}%{\mathcal{F}}

\DeclareMathOperator{\sgn}{sgn}
%\DeclareMathOperator{\erf}{erf}
\DeclareMathOperator{\erfc}{erfc}
\DeclareRobustCommand{\argmin}{\operatorname*{argmin}}
\DeclareRobustCommand{\arginf}{\operatorname*{arginf}}

\def\EE{\mathbb{E}}\def\PP{\mathbb{P}}
\def\NN{\mathbb{N}}\def\RR{\mathbb{R}}\def\ZZ{\mathbb{Z}}



\def\<{\left\langle} \def\>{\right\rangle}







\newcommand{\emphasis}[1]{\textcolor{red!80!black}{#1}}
\newcommand{\shanyin}[1]{\textcolor{blue!80!black}{#1}}

% ****************************
\begin{document}


\title{High performance computing HW4}
\author{Shanyin Tong, st3255@nyu.edu}

\maketitle

\section{Greene network test}
I tried running \texttt{pingpong} example between CPU cores on different nodes on Greene: \texttt{cs304.nyu.cluster} and \texttt{cs303.nyu.cluster}. The latency is 8.227587e-03 ms, and the bandwidth is 1.289275e+01 GB/s.

\section{MPI ring communication}
I used 4 cores and 2 nodes, i.e., 2 cores for each node on Greene:\texttt{cs043.nyu.cluster} and \texttt{cs042.nyu.cluster}. The latency is 2.438797e-03 ms, and the bandwidth is 8.051061e+00 GB/s.

\section{MPI-parallel implementation of Jacobi 2D}
I implemented the MPI-parallel implementation of Jacobi iteration for solving Poisson equation in 2D, the codes are in \texttt{Jacobi2D-mpi.cpp}, it is a hybrid of MPI and OpenMP. Here are the results for the first several iterations:
\begin{lstlisting}
Iter 0: Residual: 1023.12
Iter 10: Residual: 1019.62
Iter 20: Residual: 1017.62
Iter 30: Residual: 1016.06
Iter 40: Residual: 1014.74
Iter 50: Residual: 1013.56
Iter 60: Residual: 1012.5
Iter 70: Residual: 1011.52
Iter 80: Residual: 1010.6
Iter 90: Residual: 1009.75
Time elapsed is 20.465250 seconds.
\end{lstlisting}

\section{Plan for final project}
My plan is to implement the Biot–Savart law in 3D using CUDA, and will start with the codes implemented the law using CPU. I have no teammate yet for the project.

\end{document}